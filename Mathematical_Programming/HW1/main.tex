\documentclass[12pt, a4paper, oneside]{ctexart}
\usepackage{amsmath, amsthm, amssymb, bm, graphicx, hyperref, mathrsfs}

\title{\textbf{HW1}}
\author{方科晨}
\date{\today}
\linespread{1.5}
\newcounter{problemname}
\newenvironment{problem}{\stepcounter{problemname}\par\noindent\textbf{Problem\arabic{problemname}. }}{\\\par}
\newenvironment{solution}{\par\noindent\textbf{Solution. }}{\\\par}
% \newenvironment{note}{\par\noindent\textbf{题目\arabic{problemname}的注记. }}{\\\par}

\begin{document}

\maketitle

\begin{problem}

    \textbf{a)}设香草、薄荷、巧克力冰淇凌的量分别为 $x_1,x_2,x_3$ 加仑,令 $\bm{x}=(x_1,x_2,x_3)$\\
    令 $$A=\begin{pmatrix}
        3 & 2 & 1 & 2 & 5\\
        4 & 1.5 & 0.5 & 2 & 4\\
        3.5 & 1 & 1.5 & 1 & 3\\
    \end{pmatrix}$$ 
    $$\bm{b}=\begin{pmatrix}
        5 \\ 3 \\ 10 \\ 1.5 \\ 1
    \end{pmatrix},\\ \bm{c}=\begin{pmatrix}
        1000 \\ 500 \\ 250 \\ 480 \\ 960
    \end{pmatrix}$$ 
    由题意可得,即为求解以下线性规划问题:
    $$\begin{array}{ll}
        \textrm{maximize} & 80 \cdot \sum_{k=1}^3 x_i-\bm{x}A\bm{b}\\
        \textrm{subject to} & \bm{x}A\leq c^T\\
        & x_1\geq 50
    \end{array}$$

    \textbf{b)}使用cvx带入求解计算可得最大利润为 $10500.6$ ,其中解 $\bm{x}=(50.0000,103.3333,98.8889)$
\end{problem}

\begin{problem}
    令neighborhood $i$ 到 school $j$ , grade为 $g$ 的学生数量为 $x_{ijg}$\\
    则为求解如下线性规划问题:
    $$\begin{array}{ll}
        \textrm{minimize} & \sum_i\sum_j\sum_g x_{ijg}*d_{ij}\\
        \textrm{subject to} & \sum_i x_{ijg}\leq C_{jg},\forall j,g\\
        & \sum_j x_{ijg}=S_{ig},\forall i,g
    \end{array}$$
\end{problem}

\begin{problem}

    \textbf{a)}如果使用期望回报来求,则变为求解如下线性规划问题:
    $$\begin{array}{ll}
        \textrm{maximize} & \pi^T \bm{x}-w\cdot \bm{v}^TA\bm{x}\\
        \textrm{subject to} & \bm{x}\leq \bm{q}\\
        & \bm{x}\geq \bm{0}
    \end{array}$$
    其中 $\bm{v}$ 为离散概率分布,其余与原题设一致\\

    \textbf{b)}使用cvx可以求得最大期望获利为 $2.25$ ,\\解为 $\bm{x}^T=(10.0000,5.0000,10.0000,5.0000,5.0000)$
\end{problem}

\begin{problem}
    设 $\bm{x}=(x_1\cdots x_{12})$ ,其中 $x_i$ 表示第 $i$ 个月生产的数量,则原问题转化为如下规划问题:
    $$\begin{array}{ll}
        \textrm{minimize} & \sum_{k=1}^{12}x_i^2+s\cdot \sum_{i=1}^{12}(\sum_{k=1}^i x_i-\sum_{k=1}^i d_i)\\
        \textrm{subject to} & \sum_{k=1}^i x_i\geq \sum_{k=1}^i d_i,\forall i\\
        & x_i\leq r,\forall i
    \end{array}$$
\end{problem}

\begin{problem}
    由题意可得 $\mu=1.2,V=\begin{pmatrix}
        2 & -1\\
        -1 & 3
    \end{pmatrix}, \bm{r}=\begin{pmatrix}
        1\\2
    \end{pmatrix}$ ,且设每股在该两资产中分配的比率为 $\bm{x}$ ,则有规划问题:如下
    $$\begin{array}{ll}
        \textrm{minimize} & \bm{x}^TV\bm{x}\\
        \textrm{subject to} & \bm{r}^T\bm{x}\geq \mu \\
        & \bm{e}^T\bm{x}=1\\
        & \bm{x}\geq 0
    \end{array}$$
    将该规划问题用cvx求解可得最小的风险方差为 $0.714286$ ,解为 $\bm{x}^T=(0.5714, 0.4286)$
\end{problem}

\begin{problem}
    将每条边在出去的点出 $-1$ ,进入的点处 $+1$ ,编号 O,A,B,C,D 从 $1\sim5$ ,同时给边编号 $1\sim 7$,令矩阵
    $$A=\begin{pmatrix}
        -1 & 1 & 0 & 0 & 0 \\
        -1 & 0 & 1 & 0 & 0 \\
        0 & -1 & 0 & 1 & 0 \\
        0 & -1 & 0 & 0 & 1 \\
        0 & 1 & -1 & 0 & 0 \\
        0 & 0 & -1 & 1 & 0 \\
        0 & 0 & 0 & -1 & 1 \\
    \end{pmatrix},\bm{b}=\begin{pmatrix}
        2\\
        1\\
        1\\
        2\\
        2\\
        1\\
        1
    \end{pmatrix},\bm{c}=\begin{pmatrix}
        0.2\\
        0.2\\
        0.1\\
        0.4\\
        0.5\\
        0.2\\
        0.1
    \end{pmatrix},\bm{d}=\begin{pmatrix}
        -1\\
        0\\
        0\\
        0\\
        1
    \end{pmatrix}$$\\

    \textbf{a)}令每条边被使用的次数为 $\bm{x}^T=(x_1\cdots x_7)$ (可以大于 $1$ ,但肯定不是最优解),则变为求解如下规划问题:
    $$\begin{array}{ll}
        \textrm{minimize} & \bm{x}^T\bm{b}\\
        \textrm{subject to} & \bm{x}^T A=\bm{d}^T
    \end{array}$$

    \textbf{b)}与上同理,为如下规划问题:
    $$\begin{array}{ll}
        \textrm{maximize} & \prod_{i=1}^7 c_{i}^{x_{i}}\\
        \textrm{subject to} & \bm{x}^T A=\bm{d}^T
    \end{array}$$
\end{problem}

\begin{problem}

    \textbf{1.}代入cvx求解便可得最优值为 $1$ ,解为 $\bm{x}=(x_1,x_2,x_3)=(0.0000,0.5000,0.5000)$

    \textbf{2.}代入cvx求解便可得最优值为 $1.26795$ ,解为 $\bm{x}=(x_1,x_2,x_3)=(0.2679,0.3660,0.3660)$

    \textbf{3.}代入cvx求解便可得最优值为 $1$ ,解为 $\bm{x}=(x_1,x_2,x_3)=(0,0,1)$
\end{problem}

\end{document}