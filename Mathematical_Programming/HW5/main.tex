\documentclass[12pt, a4paper, oneside]{ctexart}
\usepackage{amsmath, amsthm, amssymb, bm, graphicx, hyperref, mathrsfs}

\title{\textbf{HW5}}
\author{方科晨}
\date{\today}
\linespread{1.5}
\newcounter{problemname}
\newenvironment{problem}{\stepcounter{problemname}\par\noindent\textbf{Problem\arabic{problemname}. }}{\\\par}
\newenvironment{solution}{\par\noindent\textbf{Solution. }}{\\\par}
\newenvironment{note}{\par\noindent\textbf{题目\arabic{problemname}的注记. }}{\\\par}

\begin{document}

\maketitle

\begin{problem}
	
	\textbf{(a)} 令 Core 集合为 $Z$ ,则 $\forall \hat z=(\hat z_1,\cdots, \hat z_n),\tilde z=(\tilde z_1,\cdots, \tilde z_n)\in Z$ ,对于任意 $\lambda\in [0,1]$ ,我们有 $\sum_{k\in K} (\lambda \hat z_k+(1-\lambda)\tilde z_k)=\lambda\sum_{k\in K}\hat z_k+(1-\lambda)\sum_{k\in K}\tilde z_k=\lambda V^K+(1-\lambda)V^K=V^K$ ,同时对于 $\forall S\subset K,\sum_{k\in S}(\lambda \hat z_k+(1-\lambda)\tilde z_k)\geq \lambda V^S+(1-\lambda)V^S=V^S$ ,所以 $\lambda \hat z+(1-\lambda)\tilde z\in Z$ ,所以 $Z$ 是凸集合。
	
	\textbf{(b)} 原问题为:
	$$\begin{array}{ll}
		\textrm{max} & c^Tx\\
		\textrm{s.t.} & A^Kx\leq b^K\\
		& x\geq 0
	\end{array}$$ 其中 $A^K,b^K$ 的定义如题。则有相应的对偶问题:
	$$\begin{array}{ll}
		\textrm{min} & (b^K)^Ty\\
		\textrm{s.t.} & (A^K)^Ty\geq c\\
		& y\geq 0
	\end{array}$$
	
	\textbf{(c)} 首先,根据题意有 $z_k=(b^{\{k\}})^Ty^*$ ,则有 $\sum_{k\in K}z_k=\sum_{k\in K}(b^{\{k\}})^Ty^*=(\sum_{k\in K}(b^{\{k\}})^T)y^*=(b^K)^Ty^*$ ,故满足第一个要求。

	其次,对于 $\forall S\subset K$ ,我们可以构造一对对偶问题,即:
	$$\begin{array}{ll}
		\textrm{min} & (b^S)^Ty\\
		\textrm{s.t.} & (A^K)^Ty\geq c\\
		& y\geq 0
	\end{array}
	$$ 和
	$$\begin{array}{ll}
		\textrm{max} & c^Tx\\
		\textrm{s.t.} & A^Kx\leq b^S\\
		& x\geq 0
	\end{array}$$
	则有 $y^*$ 是前者的可行解,且 $\sum_{k\in S}z_k=\sum_{k\in S}(b^{\{k\}})^Ty^*=(\sum_{k\in S}(b^{\{k\}})^T)y^*=(b^S)^Ty^*$ 。根据弱对偶性,我们有 $(b^S)^T y^*\geq c^Tx^*$ ,其中 $x^*$ 是后者的最优解。同时,对于集合 $S$ 我们有
	$$\begin{array}{lll}
		V^S:=&\textrm{max} & c^Tx\\
		&\textrm{s.t.} & A^Sx\leq b^S\\
		& & x\geq 0
	\end{array}$$
	根据 $A^S$ 的定义,我们有 $A^K\leq A^S$ ,其中不等号是逐元素的。所以第三个问题的最优解是第二个问题的可行解,因此第二个问题的最优值要大于等于第三个问题的最优值,即 $V^S\leq c^Tx^* \leq (b^S)^Ty^*=\sum_{k\in S}z_k$ ,故满足第二个要求。
	综上,对应的 $z$ 是在Core中。

	\textbf{(d)} 不妨考虑只有两个firm,其中一个的 $A$ 矩阵元素全部为 $+\infty$ ,但资源全部为 $1$ ,另一个的 $A$ 矩阵元素全部为 $1$ ,但资源全部为 $0$ 。那么显然有它们单独生成时的最优解均为 $0$ ,且合作生产时的最优解大于 $0$ 。因此,利润可以任意分配,均可比单独生产时优,即任意满足利润和的分配方案 $z$ , $z$ 均属于Core。

	\textbf{(e)} 

	\indent\indent \textbf{(e1)} 合并后的资源向量为 $(6;7;5)$ 可以的到最优值为 $22$ 最优解为 $x=(2;0;5)$ 。在这例中联盟会比单独生产更优,是因为单独生产时,由于资源消耗矩阵的限制,有部分资源无法被用于生产,而联盟生产时,可以使资源得到更好的利用。
	
	\indent\indent \textbf{(e2)} 大联盟的对偶问题即为:
	$$\begin{array}{ll}
		\textrm{min} & 6y_1+7y_2+5y_3\\
		\textrm{s.t.} & y_1+y_2\geq 1\\
		& y_1+y_3\geq 2\\
		& y_2+y_3\geq 4\\
		& y_1,y_2,y_3\geq 0
	\end{array}$$ 根据Complementarity Slackness可以解得最优解为 $y=(0,1,3)$ ,最优值为 $22$ 。因此可以获得Core中的一个值 $z=b^{\{k\}}y^*=(11;6;5)$ 。若单独生产的话分别最优获利 $(10;6;5)$ ,合作生产更优。
\end{problem}

\begin{problem}
	
	\textbf{(a)} 对于 $\forall x_1,x_2\in \mathbb{R}^n$ 和 $\forall \lambda\in [0,1]$ ,我们有 $f(\lambda x_1+(1-\lambda)x_2)=\sup_{c\in\mathcal{C}} c^T (\lambda x_1+(1-\lambda)x_2)\leq \sup_{c\in \mathcal{C}}c^T\lambda x_1+\sup_{c\in\mathcal{C}}c^T(1-\lambda)x_2=\lambda f(x_1)+(1-\lambda)f(x_2)$ ,故 $f$ 是凸函数。

	\textbf{(b)} 对偶问题为:
	$$\begin{array}{ll}
		\textrm{min} & d^Tg\\
		\textrm{s.t.} & F^T d=x\\
		& d\geq 0
	\end{array}$$ 。由于原问题有有界最优解 $f(x)$ ,故对偶问题有可行解。同时对偶问题的最优值也有界,故两问题的最优值相等,故对偶问题的最优值为 $f(x)$ 。

	\textbf{(c)} 根据原问题和题 \textbf{(b)} 中的表达式,我们可以得到如下线性规划问题:
	$$\begin{array}{ll}
		\textrm{min} & d^Tg\\
		\textrm{s.t.} & F^T d=x\\
		& Ax\geq b\\
		& d\geq 0
	\end{array}$$

	\textbf{(d)} 当取 $\mathcal{C}=\{c_{nom}\}$ 时代入 $(LP)$ ,解得最优值为 $1.50167$ ,相应的最优解为 $x_{nom}=\begin{pmatrix}0.697723485248259\\
	-0.628510742007053\\
	1.44824860085731\\
	-1.13155737580194\\
	0.220200744406362\\
	-0.100098786198233\\
	0.696493987316371\\
	1.25967295928702\\
	-0.752296038240486\\
	-0.323781575854262\end{pmatrix}$ 当 $\mathcal{C}$ 取题中的限制时,我们可以得到 $\mathcal{C}=\{c\in\mathbb{R}^n|Fc\leq g\}$ ,其中 $F=\begin{pmatrix}I\\-I\\e^T\\-e^T\end{pmatrix}$ 对应的 $g=\begin{pmatrix}
		1.25 c_{nom}\\
		-0.75 c_{nom}\\
		1.1 e^Tc_{nom}\\
		-0.9 e^Tc_{nom}
	\end{pmatrix}$
	将数据代入 \textbf{(c)} 中的线性规划中,可以得到最优值为 $2.31342$ ,最优解为 $x=\begin{pmatrix}
		0.109600364771020\\
		0.0238609948380699\\
		0.109600364897081\\
		0.109600364787457\\
		0.164556745013909\\
		0.109600364587373\\
		0.196988654381789\\
		0.352020173406039\\
		0.0157930690545938\\
		0.118630037159758\\
	\end{pmatrix},g=\begin{pmatrix}
		1.73478902340158\\
		1.59911631040700\\
		1.34782901038315\\
		1.71217750954077\\
		1.56740285179923\\
		2.08973018957289\\
		2.09529615163118\\
		1.89242046530937\\
		2.16076045455912\\
		2.15095977515493\\
		-1.04087341404095\\
		-0.959469786244198\\
		-0.808697406229888\\
		-1.02730650572446\\
		-0.940441711079535\\
		-1.25383811374374\\
		-1.25717769097871\\
		-1.13545227918562\\
		-1.29645627273547\\
		-1.29057586509296\\
		16.1484239327481\\
		-13.2123468540666
	\end{pmatrix}$ ,可以发现后者的最优值要大于前者,这是因为后者的 $\mathcal{C}$ 的范围更大,最坏情况包含了前者的,即还有更多的可能性,故最坏情况下的最优值要更大。
\end{problem}

\end{document}
