\documentclass[12pt, a4paper, oneside]{ctexart}
\usepackage{amsmath, amsthm, amssymb, bm, graphicx, hyperref, mathrsfs}

\title{\textbf{HW2}}
\author{方科晨}
\date{\today}
\linespread{1.5}
\newcounter{problemname}
\newenvironment{problem}{\stepcounter{problemname}\par\noindent\textbf{Problem\arabic{problemname}. }}{\\\par}
\newenvironment{solution}{\par\noindent\textbf{Solution. }}{\\\par}
\newenvironment{note}{\par\noindent\textbf{题目\arabic{problemname}的注记. }}{\\\par}

\begin{document}

\maketitle

\begin{problem}

	\textbf{(a)} $A$ 是凸集, $B$ 不是凸集。 
	$f''(x)=-2$ ,所以 $f$ 是凹函数,因此 $\forall (x_1,y_1),(x_2,y_2)\in A,\forall 0\leq \lambda\leq 1$ ,对于点 $(\lambda x_1+(1-\lambda)x_2,\lambda y_1+(1-\lambda)y_2)$ 有 $1=\lambda \cdot 1+(1-\lambda)\cdot 1\leq \lambda x_1+(1-\lambda)x_2 \leq \lambda \cdot 3+(1-\lambda)\cdot 3=3$ 且$0=\lambda \cdot 0+(1-\lambda)\cdot 0\leq \lambda y_1+(1-\lambda)y_2 \leq \lambda f(x_1)+(1-\lambda) f(x_2)\leq f(\lambda x_1+(1-\lambda)x_2)$ ,因此 $(\lambda x_1+(1-\lambda)x_2,\lambda y_1+(1-\lambda)y_2)\in A$ ,所以 $A$ 是凸集。
	
	$g''(x)=-2$ , $g$ 是凸函数,且不等式不取等,因此对于点 $(1,g(1)),(3,g(3))\in B$ ,对于 $\forall 0\leq\lambda\leq 1$ 都有 $1\leq \lambda\cdot 1+(1-\lambda)\cdot 3\leq 3$ 且 $\lambda g(1)+(1-\lambda)g(3)> g(\lambda\cdot 1+(1-\lambda)\cdot 3)$ ,因此 $(\lambda \cdot 1+(1-\lambda)\cdot 3,\lambda g(1)+(1-\lambda)g(3))\notin B$ ,故 $B$ 不是凸集。

	\textbf{(b)} $g$ 是凸函数 $f$ 不是,由 (1) 中求导可得。
\end{problem}


\begin{problem}

	$\Rightarrow$ : 若 $\Omega$ 是一个convex cone,则 $\Omega$ 既是一个锥也是一个凸集。由于 $\Omega$ 是锥,所以 $\forall x\in \Omega,\forall \lambda>0$ 都有 $\lambda x\in \Omega$ 。而且 $\forall x,y\in \Omega,\frac{1}{2} x,\frac{1}{2}y\in \Omega$ ,由于 $\frac{1}{2}+\frac{1}{2}=1$ 且 $\Omega$ 为凸集,所以 $\frac{1}{2}x+\frac{1}{2}y\in \Omega$ ,故 $x+y=2(\frac{1}{2}x+\frac{1}{2}y)\in \Omega$

	$\Leftarrow$ : 由于 $\forall x\in \Omega,\forall \lambda>0$ 都有 $\lambda x\in \Omega$ ,所以 $\Omega$ 是一个锥。其次 $\forall x,y\in \Omega,\forall 0\leq \lambda \leq 1$ 。当 $\lambda=0,\frac{1}{2},1$ 时是trival的。当 $0\leq \lambda < \frac{1}{2}$ 时,有 $\lambda x+(1-\lambda)y=\lambda(x+y)+(1-2\lambda)y$ ,其中由于 $x,y\in\Omega$ 故 $x+y\in\Omega$ 故 $\lambda(x+y)\in \Omega$ ,又 $1-2\lambda>0$ 则 $(1-2\lambda)y\in \Omega$,两者相加可得 $\lambda(x+y)+(1-2\lambda)y\in \Omega$ 。当 $\frac{1}{2}<\lambda<1$ 时同理可得。综上, $\Omega$ 是凸集,故 $\Omega$ 是convex cone。
\end{problem}

\begin{problem}

	首先证明 $\partial f(\hat x)$ 是凸集。考虑任意 $g_1,g_2\in \partial f(\hat x)$ 以及任意 $0\leq \alpha\leq 1$ ,则有 $f(x)=\alpha f(x)+(1-\alpha)f(x)\geq \alpha(f(\hat x)+g_1^T(x-\hat x))+(1-\alpha)(f(\hat x)+g_2^T(x-\hat x))=(\alpha+(1-\alpha))f(\hat x)+(\alpha g_1^T+(1-\alpha)g_2^T)(x-\hat x)=f(\hat x)+(\alpha g_1^T+(1-\alpha)g_2^T)(x-\hat x)$ 。故有 $\alpha g_1^T+(1-\alpha)g_2^T\in \partial f(\hat x)$ 。因此 $\partial f(\hat x)$ 是凸集。
	
	由于 $\partial f(\hat x)$ 中的条件为等号,所以显然 $\partial f(\hat x)$ 为闭集,任意 $\partial f(\hat x)$ 中点列的极限仍然在 $\partial f(\hat x)$ 中。

	$f$ 是凸函数,则 $\textrm{epi}f$ 是凸集,且 $(\hat x,f(\hat x))\in \partial \mathrm{epi}f$ 。由支撑超平面,可以找到 $(\bm{a},b)\neq \bm{0},(\bm{a},b)\in \mathbb{R}^{n+1}$ 使得 $(\bm{a},b)^T(x,f(x))\geq (\bm{a},b)^T(\hat x,f(\hat x)),\forall x\in \mathbb{R}^n$ 。如果 $b=0$ ,则式子变为 $\bm{a}\cdot x\geq \bm{a}\cdot \hat x,\forall x\in\mathbb{R}^n$ ,这显然不可能成立,故 $b\neq 0$ 。则原式两边同除以 $b$ 就有 $(\frac{\bm{a}}{b},1)^T(x,f(x))\geq (\frac{\bm{a}}{b},1)^T(\hat x,f(\hat x)),\forall x\in \mathbb{R}^n$ ,由此可见 $(\frac{\bm{a}}{b},1)\in \partial f(\hat x)$ ,因此 $\partial f(\hat x)$ 非空。

	综上, $\partial f(\hat x)$ 为非空闭凸集。
\end{problem}

\begin{problem}
	
	考虑 $x,y\in \bar C$ ,则存在两个点列 $\{x_n\}_{n=1}^\infty\subset C,\{y_n\}_{n=1}^\infty\subset C$ 使得 $\lim_{n\to \infty}x_n=x,\lim_{n\to \infty}y_n=y$ 。则对于任意 $0\leq \alpha\leq 1$ 。由于 $C$ 为凸集,故有 $z_n=\alpha x_n+(1-\alpha)y_n\in C,\forall n\geq 1$ ,则 $z=\alpha x+(1-\alpha)y=\alpha \lim_{n\to \infty}x_n+(1-\alpha)\lim_{n\to\infty}y_n=\lim_{n\to\infty}(\alpha x_n+(1-\alpha)y_n)\in\bar C$ 。因此 $\bar C$ 为凸集。
\end{problem}

\begin{problem}
	
	\textbf{(a)} 对于任意的 $0\leq \alpha\leq 1$ 和 $x,y\in\mathbb{R}^n$ 有 $h(\alpha x+(1-\alpha)y)=g(f(\alpha x+(1-\alpha)y))$ 。由于 $f(\alpha x+(1-\alpha)y)\leq \alpha f(x)+(1-\alpha)f(y)$ 及 $g$ 为单调非降函数,故 $h(\alpha x+(1-\alpha)y)\leq g(\alpha f(x)+(1-\alpha)f(y))\leq \alpha g(f(x))+(1-\alpha)g(f(y))=\alpha h(x)+(1-\alpha) h(y)$ ,所以 $h$ 是凸函数。

	\textbf{(b)} 令 $f,g:\mathbb{R}\to\mathbb{R},f(x)=x^2-1,g(x)=x^2,h=g\circ f$ 。我们有 $h(0)=g(f(0))=g(-1)=1,h(1)=g(f(1))=g(0)=0,h(\frac{1}{2})=g(f(\frac{1}{2}))=g(-\frac{3}{4})=\frac{9}{16}$ 。可以发现并不满足 $h(\frac{1}{2}\cdot 0+\frac{1}{2}\cdot 1)\leq \frac{1}{2}h(0)+\frac{1}{2}h(1)$ ,故 $h$ 不为凸函数。
\end{problem}

\begin{problem}
	
	由于 $\lambda_1>0$ ,则有 $-\frac{\lambda_2}{\lambda_1},\cdots,-\frac{\lambda_n}{\lambda_1},\frac{1}{\lambda_1}=\frac{\lambda_1+\cdots,\lambda_n}{\lambda_1}>0$ ,同时 $(-\frac{\lambda_2}{\lambda_1})+\cdots+(-\frac{\lambda_n}{\lambda_1})+(\frac{\lambda_1+\cdots+\lambda_n}{\lambda_1}) = 1$ ,由于 $f$ 为凸函数,由Jensen不等式可得 $f(x_1)=f(\frac{\lambda_1}{\lambda_1})\leq -\frac{\lambda_2}{\lambda_1}f(x_2)+\cdots-\frac{\lambda_n}{\lambda_1}f(x_n)+\frac{1}{\lambda_1}f(\lambda_1x_1+\cdots+\lambda_n x_n)$ ,整理后即为要证的式子
\end{problem}

\begin{problem}
	
	可以求得 $f$ 的Hessian矩阵为:

	$$\nabla^2 f(x)=\{\frac{\partial^2 f}{\partial x_i\partial x_j}\}=
	\begin{pmatrix}
		\frac{\lambda_1 e^{-x_1}}{1-e^{-x_1}}\frac{\lambda_1e^{-x_1}-1}{1-e^{-x_1}}f(x) & \frac{\lambda_1 e^{-x_1}}{1-e^{-x_1}}\frac{\lambda_2e^{-x_2}}{1-e^{-x_2}}f(x) & \cdots & \frac{\lambda_1 e^{-x_1}}{1-e^{-x_1}}\frac{\lambda_n e^{-x_n}}{1-e^{-x_n}}f(x)\\
		\frac{\lambda_2 e^{-x_2}}{1-e^{-x_2}}\frac{\lambda_1e^{-x_1}}{1-e^{-x_1}}f(x) & \frac{\lambda_2 e^{-x_2}}{1-e^{-x_2}}\frac{\lambda_2e^{-x_2}-1}{1-e^{-x_2}}f(x) & \cdots & \frac{\lambda_2 e^{-x_2}}{1-e^{-x_2}}\frac{\lambda_n e^{-x_n}}{1-e^{-x_n}}f(x)\\
		\vdots & \vdots & \ddots & \vdots\\
		\frac{\lambda_n e^{-x_n}}{1-e^{-x_n}}\frac{\lambda_1 e^{-x_1}}{1-e^{-x_1}}f(x) & \frac{\lambda_2 e^{-x_2}}{1-e^{-x_2}}\frac{\lambda_n e^{-x_n}}{1-e^{-x_n}}f(x) & \cdots & \frac{\lambda_n e^{-x_n}}{1-e^{-x_n}}\frac{\lambda_n e^{-x_n}-1}{1-e^{-x_n}}f(x)\\
	\end{pmatrix}$$

	不妨令 $y=(\hat y_1,\cdots,\hat y_n)=((1-e^{-x_1})y_1,\cdots,(1-e^{-x_n})y_n)\in \mathbb{R}^n$ ,其中 $1-e^{-x_i}>0,\forall i=1,\cdots,n$ 。再设 $c_i=\lambda_i e^{-x_i}$ 
	,因此 $\hat y^T\nabla^2f(x)\hat y=f(x)\cdot(\sum_{i=1}^n c_i(c_i-1)y_i^2+2\cdot \sum_{i\neq j}c_ic_jy_iy_j)=f(x)\cdot(-\sum_{i=1}^nc_iy_i^2+(\sum_{i=1}^nc_iy_i)^2)(*)$
	
	由Cauchy-Schwarz不等式以及 $f(x)>0,\forall x\in\mathbb{R}^n$ 以及题设 $\sum_{i=1}^nc_i\leq 1$,可得 $(*)\leq f(x)\cdot (-\sum_{i=1}^n c_iy_i^2+(\sum_{i=1}^nc_i)(\sum_{i=1}^nc_iy_i^2))\leq f(x)\cdot(-\sum_{i=1}^nc_iy_i^2+\sum_{i=1}^nc_iy_i^2)=0$ 。
	
	综上 $f$ 是凹函数。
\end{problem}
\end{document}
