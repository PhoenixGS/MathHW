\documentclass[12pt, a4paper, oneside]{ctexart}
\usepackage{amsmath, amsthm, amssymb, bm, graphicx, hyperref, mathrsfs}

\title{\textbf{week3}}
\author{方科晨}
\date{\today}
\linespread{1.5}
\newcounter{problemname}
\newenvironment{problem}{\stepcounter{problemname}\par\noindent\textbf{Problem. }}{\\\par}
\newenvironment{solution}{\par\noindent\textbf{Solution. }}{\\\par}
\newenvironment{note}{\par\noindent\textbf{题目\arabic{problemname}的注记. }}{\\\par}

\begin{document}

\maketitle

\section{Chatper 2}
\begin{problem} 4

	$f(x)=x^3-a=0$ ,则 $f(x)$ 的根即为 $\sqrt[3]{a}$ 。故有Newton迭代法 $x_{k+1}=x_k-\frac{f(x_k)}{f'(x_k)}=x_k-\frac{x_k^3-a}{3\cdot x_k^2}$ 。

	考虑局部收敛性, $f(x)$ 在 $x^*=\sqrt[3]{a}$ 附近有连续二阶导数,故至少二阶收敛。考虑 $g''(x^*)=\frac{f''(x^*)}{f'(x^*)}\neq 0$ ,故该牛顿法为二阶收敛。
\end{problem}

\begin{problem} 5

	首先当 $x$ 在 $\sqrt{a}$ 附近时, $x=\frac{x(x^2+3a)}{3x^2+a}\Leftrightarrow 3x^3+ax=x^3+3ax\Leftrightarrow x=\sqrt{a}$ ,故改迭代方法的确是收敛于 $\sqrt{a}$\\

	可求得 $\lim_{k\to\infty}\frac{\sqrt{a}-x_{k+1}}{(\sqrt{a}-x_k)^3}=\lim_{k\to\infty}\frac{\frac{x_k(x_k^2+3a)}{3x_k^2+a}-\sqrt{a}}{(x_k-\sqrt{a})^3}=\lim_{k\to\infty}\frac{x_k(x_k^2+3a)-\sqrt{a}(3x_k^2+a)}{(x_k-\sqrt{a})^3(3x_k^2+a)}=\lim_{k\to\infty}\frac{x_k^3-3\sqrt{a}x_k^2+3ax_k-a\sqrt{a}}{(x_k^3-3\sqrt{a}x_k^2+3ax_k-a\sqrt{a})(3x_k^2+a)}=\lim_{k\to\infty}\frac{1}{3x_k^2+a}=\frac{1}{4a}$ 为常数,故是三阶收敛的
\end{problem}

\begin{problem} 6
	
	由 $\varphi'=\frac{ff''}{(f')^2}$ 直接计算 $\varphi''=\frac{(ff'')'(f')^2-ff''((f')^2)'}{(f')^4}=\frac{f'f'f'f''+f'f'ff'''-2ff'f''f''}{(f'^4)}$ ,由于 $f'(x^*)\neq 0,f(x^*)=0$ ,故有 $\varphi''(x^*)=\frac{(f'(x^*))^3f''(x^*)}{(f'(x^*))^4}=\frac{f''(x^*)}{f'(x^*)}$
\end{problem}

\begin{problem} 9
	
	\textbf{(1)} $f(x)=x^3-3x-1$ ,则有 $f'(x)=3x^2-3$ ,牛顿迭代公式为 $x_{k+1}=x_k-\frac{x_k^3-3x_k-1}{3x_k^2-3}$ ,前几次迭代结果为 $x_0=2,x_1=1.8888888888888888,x_2=1.879451566951567$ 已符合精度要求。

	\textbf{(2)} 直接按割线法计算公式,可求得前几次迭代结果为 $x_0=2.0,x_1=1.9,x_2=1.8810939357907253,x_3=1.8794110601699177$ 符合精度要求
\end{problem}

\section{Chatper 3}

\begin{problem} 2
	
	由于 $\Vert x\Vert_\infty=\max_{1\leq i\leq n}|x_i|$ ,因此有 $\Vert x\Vert_\infty=\max_{1\leq i\leq n}|x_i|\leq \sum_{i=1}^n |x_i|=\Vert x\Vert_1\leq \sum_{i=1}^n(\max_{1\leq i\leq n}|x_i|)=n \max_{1\leq i\leq n}|x_i|=n\Vert x\Vert_\infty$
\end{problem}

\end{document}
